\begin{problem}{Fucs và gánh nặng mưu sinh}{standard input}{standard output}{5 seconds}{1024 megabytes}

Server của Fucs nhận được một hàng đợi gồm $N$ tác vụ cần xử lý. Vì tài nguyên CPU là hữu hạn nên chỉ có thể xử lý tuần tự từng tác vụ.

Một khi CPU đã bắt đầu chạy một tác vụ, nó phải xử lý cho đến khi tác vụ đó hoàn thành \textbf{(nguyên tắc Atomic)}, không được phép ngắt quãng để chuyển sang tác vụ khác.

Mỗi tác vụ $i$ đi kèm các thông tin:
\begin{itemize}
    \item Thời gian chạy: $r_i$
    \item Hạn hoàn thành (Deadline): $t_i$
    \item Lợi nhuận nếu hoàn thành đúng hạn: $p_i$
\end{itemize}

\textbf{Quy tắc:}
Bạn có quyền chọn làm hoặc không làm một tác vụ. Nếu chọn làm, tác vụ phải hoàn thành trước hoặc đúng thời điểm $t_i$. Nếu hoàn thành sau thời điểm $t_i$, tác vụ đó coi như thất bại và không mang lại lợi nhuận ($p_i = 0$).

\textbf{Nhiệm vụ:}
Hãy lập trình chọn và sắp xếp thứ tự thực hiện các tác vụ sao cho Fucs thu được \textbf{tổng lợi nhuận lớn nhất có thể}.

\InputFile
\begin{itemize}
    \item Dòng đầu tiên chứa số nguyên dương $N$.
    \item $N$ dòng tiếp theo, mỗi dòng chứa 3 số nguyên $r_i, t_i, p_i$ tương ứng với thời gian chạy, deadline và lợi nhuận của tác vụ thứ $i$.
\end{itemize}

\OutputFile
\begin{itemize}
    \item Dòng đầu tiên: Ghi một số nguyên là tổng lợi nhuận lớn nhất tìm được.
    \item Dòng thứ hai: Ghi dãy các chỉ số (ID) của các tác vụ được chọn theo thứ tự thực hiện. (ID tính từ 1 đến $N$).
\end{itemize}

\Scoring
\textbf{Subtask 1 (Chiếm 0\% số điểm): AC subtask này là điều kiện đề tính điểm subtask2}
\begin{itemize}
    \item $N \le 20$
    \item $1 \le r_i, t_i, p_i \le 10^5$
    \item \textbf{Yêu cầu:} Bắt buộc tìm ra nghiệm \textbf{tối ưu tuyệt đối}.
    \item \textbf{Cách tính điểm:} Chấm chính xác (Correct/Incorrect).
\end{itemize}

\textbf{Subtask 2 (Chiếm 100\% số điểm):}
\begin{itemize}
    \item $N \le 500$
    \item $1 \le r_i, t_i, p_i \le 10^{12}$
    \item \textbf{Yêu cầu:} Tìm nghiệm \textbf{gần đúng tốt nhất}.
    \item \textbf{Cách tính điểm:} Điểm số được tính dựa trên độ lệch so với đáp án của Ban ra đề (Jury) theo công thức mũ:
    $ \text{Score} = min(1, \left(\frac{YourAnswer}{MyAnswer}\right)^{3.6}) $
    \textit{(Hệ số mũ 3.6 được sử dụng một cách ngẫu nhiên).}
\end{itemize}

\Example

\begin{example}
\exmpfile{example.01}{example.01.a}%
\end{example}

\Note
Giải thích: 50 + 100 + 200 = 350.

\end{problem}

