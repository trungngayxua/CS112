Cục điều phối bộ nhớ của Bộ Quốc Phòng đang thử nghiệm chip giám sát "Fenrir". Fenrir được hàn thẳng vào bus bộ nhớ và tự động duy trì \emph{các tổng tiền tố} cho từng cột trong lưới nhớ. Lớp firmware bảo mật của Fenrir chỉ cung cấp đúng hai lệnh phần cứng: \texttt{inc(c, k)} (cộng thêm $k$ tiến trình vào cột $c$) và \texttt{pref(c)} (trả về tổng số tiến trình hiện diện trong các cột từ $1$ tới $c$). Bất kỳ thuật toán nào đòi hỏi truy cập ngẫu nhiên hay cấu trúc dữ liệu khác (ví dụ segment tree) đều bị khóa và báo lỗi "không tương thích giao diện". Nghĩa là mọi xử lý số lượng tiến trình trên các đoạn cột liên tiếp đều buộc phải ghép lời giải vào các lệnh Fenrir -- chính là Fenwick tree.

Bộ nhớ vật lý của hệ thống là một lưới $H \times W$ (đánh số hàng, cột từ $1$). Có $n$ tiến trình đã chiếm các ô $(r_i, c_i)$ và danh sách này được cập nhật liên tục qua chuẩn Fenrir. Mỗi đêm, Tổng trạm yêu cầu báo cáo "phân mảnh": hãy đặt đúng một đường cắt ngang giữa hai hàng liên tiếp và một đường cắt dọc giữa hai cột liên tiếp để chia lưới thành bốn miền chữ nhật. Chi phí của cấu hình là số tiến trình lớn nhất nằm trong một miền (họ sẽ chuyển tiến trình ra khỏi miền "đông dân nhất" để tái cân bằng). Nhiệm vụ của bạn là chọn vị trí hai đường cắt sao cho chi phí nhỏ nhất có thể, tuân thủ ràng buộc truy vấn/ cập nhật đúng định dạng Fenrir.
